\documentclass[openany, oneside,11pt]{book}
\usepackage[
  top=0.65in,
  bottom=0.25in,
  left=1.25in,
  right=1.25in,
  bindingoffset=0.25in,
  heightrounded
]{geometry}
\usepackage{listings}
\usepackage{underscore}
\usepackage{graphicx}
\usepackage{longtable}
\usepackage{float}
\usepackage{fancyhdr}
\usepackage[bookmarks=true]{hyperref}
\usepackage[utf8]{inputenc}
\usepackage[english]{babel}
\usepackage{multirow}
\usepackage[normalem]{ulem}
\useunder{\uline}{\ul}{}
\usepackage{fancyhdr}
\pagestyle{plain}
\renewcommand{\chaptermark}[1]{\markboth{#1}{}}
\fancyhf{}
\fancyhead[RE]{\chaptername~\thechapter}
\fancyhead[LO]{\leftmark}
\fancyhead[HR]{B-Track SRS}
\fancyhead[CH]{Page \thepage}
\hypersetup{
    bookmarks=false,    % show bookmarks bar?
    pdftitle={Software Requirement Specification},    % title
    pdfauthor={Albert Negura},                     % author
    pdfsubject={TeX and LaTeX},                        % subject of the document
    pdfkeywords={TeX, LaTeX, graphics, images}, % list of keywords
    colorlinks=true,       % false: boxed links; true: colored links
    linkcolor=blue,       % color of internal links
    citecolor=black,       % color of links to bibliography
    filecolor=black,        % color of file links
    urlcolor=purple,        % color of external links
    linktoc=page            % only page is linked
}%
\def\myversion{1.0 }
\date{}
%\title
\addto\captionsenglish{\renewcommand{\chaptername}{Section}}
\usepackage{hyperref}
\begin{document}
\thispagestyle{empty}
\begin{flushright}
    \rule{16cm}{5pt}\vskip1cm
    \begin{bfseries}
        \Huge{SOFTWARE REQUIREMENTS\\ SPECIFICATION}\\
        \vspace{1.5cm}
        for\\
        \vspace{1.5cm}
        B-Track\\
        \vspace{1.5cm}
        \LARGE{Version \myversion}\\
        \vspace{1.5cm}
        Prepared by Albert Negura\\
        \vspace{1.5cm}
        \today\\
    \end{bfseries}
\end{flushright}
\clearpage
\tableofcontents
\setcounter{page}{0}
\chapter{Document Overview}
\thispagestyle{fancy}
This document outlines the SRS including the complete project specifications, requirements, complete database specifications and development plan for B-Track, a Bug Tracker or Issue Tracker or Ticket system web app developed in $C\#$, employing the $.NET$ framework, $MVC$ pattern and $SQL$ Server Database technology. This document was written and compiled in LaTeX.
\newline
\newline
Section 2 provides an overview of the project, the indended stakeholders and the assumptions and constraints used in the development of this project in a way meant to mirror the method applied by various corporations for similar software development projects.
\newline
\newline
Section 3 provides a list, description and purpose for all the functional and non-functional requirements relating the end user requirements, user interface, performance, security and software quality.
\newline
\newline
Section 4 provides a complete classification and ranking of all the functional requirements while also providing the dependencies for each of the functional requirements.
\newline
\newline
Section 5 provides the corresponding Database design, basic UI design and UML design, together with other explanatory diagrams.
\newline
\newline
The necessary planning and other helpful material not directly corresponding to the software requirements specifications will be featured in the Appendix.
\newline
\newline


\chapter{Project Overview}
\thispagestyle{fancy}

\section{Project Introduction}
B-Track is a Bug Tracker, Issue Tracker or Ticket system meant to mirror the functionality of similar modern day system used by various corporations throughout the world. Such a system is used to maintain accurate records of development processes on a per-project basis or to help with customer or client tech support depending on the desired use case. 
\section{Stakeholders}

This project does not have any direct clients and was created with the goal to act as a suitable programming exercise and example of good software development practices.
\section{Assumptions and Constraints}

This project is meant to act as a Web-based service application developed using the JetBrains Rider 2020 IDE Professional edition and the JetBrains DataGrip 2020 database IDE, available through the JetBrains student package. It also employs the Model-View Controller design pattern at its core. The web pages are to be built with HTML5 and Bootstrap. The page should be hosted online using AWS, with the link being provided on GitHub.


\chapter{Requirements}
\thispagestyle{fancy}
This section describes all the functional and non-functional requirements and / or features for the Bug Tracking software. The project will only be considered complete upon having all the requirements outlined in this section met.

\section{Functional Requirements}
\subsection{End User Requirements}
End users must be able to perform the following actions:
\begin{table}[]
\begin{tabular}{|l|l|l|}
\hline
   & Description                                                                                                                                                                & Purpose                                                                                                           \\ \hline
1  & Register as a user / Login                                                                                                                                                 & Secure system access                                                                                              \\ \hline
2  & Assign/unassign users to/from roles                                                                                                                                        & Role-based security                                                                                               \\ \hline
3  & Create projects                                                                                                                                                            & \begin{tabular}[c]{@{}l@{}}Organization of resources / project managers or\\ admin only\end{tabular}              \\ \hline
4  & Assisng/unassign users to/from projects                                                                                                                                    & \begin{tabular}[c]{@{}l@{}}Organization of resources / project managers or\\ admin only\end{tabular}              \\ \hline
5  & Create tickets                                                                                                                                                             & Log software issue instance                                                                                       \\ \hline
6  & Assign tickets to user                                                                                                                                                     & \begin{tabular}[c]{@{}l@{}}Assign responsible developer to ticket and\\ enforce accountability\end{tabular}       \\ \hline
7  & Edit submitted tickets                                                                                                                                                     & Make modifications to existing tickets                                                                            \\ \hline
8  & Create ticket comments                                                                                                                                                     & \begin{tabular}[c]{@{}l@{}}Add progress comments and other important\\ information to ticket history\end{tabular} \\ \hline
9  & Create ticket attachments                                                                                                                                                  & \begin{tabular}[c]{@{}l@{}}Add helpful visuals and other documentation \\ to the ticket history\end{tabular}      \\ \hline
10 & List comments, attachments per ticket                                                                                                                                      & Organization of resources                                                                                         \\ \hline
11 & List history of changes to ticket                                                                                                                                          & \begin{tabular}[c]{@{}l@{}}Very important view for the developer on the\\ project\end{tabular}                    \\ \hline
12 & List tickets by owner                                                                                                                                                      & \begin{tabular}[c]{@{}l@{}}Provide easy tracking of tickets logged by a\\ particular user\end{tabular}            \\ \hline
13 & List tickets by assignment                                                                                                                                                 & \begin{tabular}[c]{@{}l@{}}Provide easy tracking of tickets logged by a\\ particular developer\end{tabular}       \\ \hline
14 & List tickets by project                                                                                                                                                    & Organization of resources                                                                                         \\ \hline
15 & \begin{tabular}[c]{@{}l@{}}Filter ticket lists by ticket type, priority\\ and status\end{tabular}                                                                          & Ease of use and access                                                                                            \\ \hline
16 & \begin{tabular}[c]{@{}l@{}}Filter ticket lists by creation date/time\\ (i.e. all tickets after indicated time/date)\end{tabular}                                           & Ease of use and access                                                                                            \\ \hline
17 & \begin{tabular}[c]{@{}l@{}}Sort ticket list by title, owner,\\ assignment, creation of recent update\\ date/time, ticket type, priority, status\\ and project\end{tabular} & Ease of use and access                                                                                            \\ \hline
18 & Full text search of all relevant fields                                                                                                                                    & Ease of use and access                                                                                            \\ \hline
\end{tabular}
\end{table}
\thispagestyle{fancy}
\subsection{User Interface Requirements}
To facilitate intuitive and user-friendly interaction, the user interface must be designed according to the following requirements:
\begin{longtable}{|c|c|l|}
\hline
                   & Page Description                    & \multicolumn{1}{c|}{Feature List}                                                                                                                                                                                                                                                                                                                                                                     \\ \hline
\multirow{2}{*}{1} & \multirow{2}{*}{Landing}            & 1) All users are redirected to this page when accessing the site.                                                                                                                                                                                                                                                                                                                                     \\ \cline{3-3} 
                   &                                     & \begin{tabular}[c]{@{}l@{}}2) Links to Login, Register and Demo (demo mode should\\ grant admin privileges to the user for demonstration purposes)\end{tabular}                                                                                                                                                                                                                                       \\ \hline
\multirow{2}{*}{2} & \multirow{2}{*}{Login / Register}   & \begin{tabular}[c]{@{}l@{}}1) Users can login, register as a new user and recover lost\\ password\end{tabular}                                                                                                                                                                                                                                                                                        \\ \cline{3-3} 
                   &                                     & 2) Include Social Networks as a registration / login method                                                                                                                                                                                                                                                                                                                                           \\ \hline
\multirow{3}{*}{3} & \multirow{3}{*}{Dashboard}          & \begin{tabular}[c]{@{}l@{}}1) Users are redirected to the Dashboard upon a successful\\ login operation\end{tabular}                                                                                                                                                                                                                                                                                  \\ \cline{3-3} 
                   &                                     & \begin{tabular}[c]{@{}l@{}}2) Display most recent ticket assignments or tickets logged\\ or other relevant role-based information for the \\ corresponding role of the user\end{tabular}                                                                                                                                                                                                              \\ \cline{3-3} 
                   &                                     & \begin{tabular}[c]{@{}l@{}}3) Display data visualization of ticket types logged and\\ resolved per project\end{tabular}                                                                                                                                                                                                                                                                               \\ \hline
4                  & Role Management                     & \begin{tabular}[c]{@{}l@{}}1) Admin users must have access to an interface to assign\\ users to and remove users from the following roles:\\ - Administrator (Explanation: manages the system)\\ - Project Manager (Explanation: manages developer and ticket\\   assignment)\\ - Developer (Explanation: responsible for resolving tickets)\\ - User (Explanation: submits the tickets)\end{tabular} \\ \hline
\multirow{7}{*}{5} & \multirow{7}{*}{Ticket Management}  & 1) Paged listing of ticket items (default: 10 tickets per page)                                                                                                                                                                                                                                                                                                                                       \\ \cline{3-3} 
                   &                                     & \begin{tabular}[c]{@{}l@{}}2) Users have access to ticket lists based on role:\\ - Users see tickets which they own\\ - Developers see tickets to which they are assigned\\ - Project managers see tickets belonging to the projects for\\   which they are responsible\\ - Administrators see all tickets\end{tabular}                                                                               \\ \cline{3-3} 
                   &                                     & \begin{tabular}[c]{@{}l@{}}3) Tickets may be created and edited, not deleted. Completed\\ tickets receive a "Resolved" status.\end{tabular}                                                                                                                                                                                                                                                           \\ \cline{3-3} 
                   &                                     & \begin{tabular}[c]{@{}l@{}}4) A ticket details page must be used to provide full detailed\\ detailed ticket history for each ticket.\end{tabular}                                                                                                                                                                                                                                                     \\ \cline{3-3} 
                   &                                     & \begin{tabular}[c]{@{}l@{}}5) Ticket statuses, types, and priorities should be seeded in\\ database.\end{tabular}                                                                                                                                                                                                                                                                                     \\ \cline{3-3} 
                   &                                     & 6) Provide interfaces for adding comments and attachments.                                                                                                                                                                                                                                                                                                                                            \\ \cline{3-3} 
                   &                                     & \begin{tabular}[c]{@{}l@{}}7) Developers should be notified via email or other messaging\\ service when they are assigned a new ticket.\end{tabular}                                                                                                                                                                                                                                                  \\ \hline
\multirow{2}{*}{6} & \multirow{2}{*}{Project Management} & \begin{tabular}[c]{@{}l@{}}1) Project managers and administrators must have access to an\\ interface for creating projects.\end{tabular}                                                                                                                                                                                                                                                              \\ \cline{3-3} 
                   &                                     & \begin{tabular}[c]{@{}l@{}}2) Project managers and administrators must have access to an\\ interface for assigning a user to and removing a user from a\\ project.\end{tabular}                                                                                                                                                                                                                       \\ \hline
\multirow{3}{*}{7} & \multirow{3}{*}{User Profile}       & 1) Users must have the ability to change their password.                                                                                                                                                                                                                                                                                                                                              \\ \cline{3-3} 
                   &                                     & \begin{tabular}[c]{@{}l@{}}2) Users should be able to edit their profile information\\ (name, email, etc.).\end{tabular}                                                                                                                                                                                                                                                                              \\ \cline{3-3} 
                   &                                     & \begin{tabular}[c]{@{}l@{}}3) Users should be able to associate their profiles to social \\ network identities.\end{tabular}                                                                                                                                                                                                                                                                          \\ \hline
\end{longtable}
\thispagestyle{fancy}
\section{Non-functional Requirements}

\subsection{Performance Requirements}
\subsection{Security Requirements}
\subsection{Software Quality Attributes}

\thispagestyle{fancy}
\chapter{Classification of Functional Requirements}
\section{End User Requirements}
\section{User Interface Requirements}

\thispagestyle{fancy}
\chapter{Developer Add-ons}
\section{Database Design}
\section{User Interface Design}
\section{UML Design}

\appendix
\end{document}
